\section{Errors}
%The values of the x(i); y(i)'s are measured with a camera, and there is a
%measurement uncertainty that is estimated to 1mm on each coordinate.
%Estimate the resulting errors on the found parameters using NR, Eq.
%(15.4.19).
The deviation can be found using equation \ref{eq:stdDeviation}.

\begin{equation}
\sigma\left(a_j\right) = \sqrt{ \sum_{i=0}^{N-1} \left( \left( \frac{V_{ji}}{w_i}\right)^2\right) }
\label{eq:stdDeviation}
\end{equation}

This is implemented in code segment \ref{code:deviation}.
Because of the linear dependency, then the division of \texttt{V} by \texttt{w} would result in infinity when calculating the standard deviation.
To avoid this, all \texttt{V}'s which are divided by a \texttt{w} which equals zero, because it is less than the threshold, is set to zero before hand (line 73 to 75, code segment \ref{code:deviation}).


\lstinputlisting[firstnumber=69,
        firstline=69, 
        lastline=79,
        caption={Computing the standard deviation.},
        label={code:deviation}
        ]{../nr_robot/main.cpp}

This results in the standard deviation seen in equation \ref{eq:stdDeviation}.

\begin{equation}
\sigma_{d1}\left(a_j\right) = 
    \left[
        \begin{array}{c}
        0.044846 \\
        0.044781 \\
        0.044808 \\
        0.044833
        \end{array}
    \right]
\qquad
\sigma_{d2}\left(a_j\right) = 
    \left[
        \begin{array}{c}
        0.029733 \\
        0.040244 \\
        0.022375 \\
        0.044783
        \end{array}
    \right]
\label{eq:stdDeviation}
\end{equation}