\section{Errors}
%The values of the x(i); y(i)'s are measured with a camera, and there is a
%measurement uncertainty that is estimated to 1mm on each coordinate.
%Estimate the resulting errors on the found parameters using NR, Eq.
%(15.4.19).
The deviation can be found using equation \ref{eq:stdDeviation}.
\[
\sigma\left(a_j\right) = \sqrt{ \sum_{i=0}^{N-1} \left( \left( \frac{V_{ji}}{w_i}\right)^2\right) }
\label{eq:stdDeviation}
\]
This is implemented in \lstlistingname~\ref{code:deviation}.
\lstinputlisting[firstnumber=63,
        firstline=63, 
        lastline=70,
        caption={Computing the standard deviation.},
        label={code:deviation}
        ]{../nr_robot/main.cpp}

This results in the standard deviation seen in equation \ref{eq:stdDeviation} for the first dataset (d1).

\begin{equation}
\sigma\left(a_j\right) = 
\left[
\begin{array}{c}
 0.044846 \\
 0.044781 \\
 0.044808 \\
 0.044833
\end{array}
\right]
\label{eq:stdDeviation}
\end{equation}