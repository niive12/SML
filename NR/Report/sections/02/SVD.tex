\section{The SVD Decomposition}
%Read the les with values, insert them in A and compute U, W og
%V using the method from NR. State (with arguments based on the
%SVD matrices) if there are linear dependencies between the parameters
%x0; y0; a; b.

$A$ and $z$ is constructed using code \ref{code:computing_A_z}, this results in the format for $A$ and $z$ as seen in equation \ref{eq:mathematicalModel}.

\lstinputlisting[firstnumber=32,
        firstline=32, 
        lastline=50,
        caption={Computing $A$ and $z$.},
        label={code:computing_A_z}
        ]{../nr_robot/main.cpp}

When computing the SVD using the library supplied by Numerical Recipes, the matrix in equation \ref{eq:W_and_V_computed} for $W$
% and $V$ are 
is gained.

\begin{equation}
%W
W_{diag_{d1}} = 
\left[
\begin{array}{c}
23.3788 \\
22.6835 \\
21.8248 \\
21.5074
\end{array}
\right]
% 
\qquad
% 
W_{diag_{d2}} = 
\left[
\begin{array}{c}
31.6427\\
22.7755\\
21.9092\\
0.0456291
\end{array}
\right]
%\qquad
%V = 
%\left[
%\begin{array}{c c c c}
%0.589669 & -0.354287 & -0.351603 & 0.634938 \\
%0.311263 & 0.703527 & 0.509408 & 0.385577 \\
%0.493938 & -0.470162 & 0.629460 & -0.372496 \\
%0.558060 & 0.398095 & -0.469743 & -0.556265
%\end{array}
%\right]
\label{eq:W_and_V_computed}
\end{equation}

Where $W_{diag}$ in equation \ref{eq:W_and_V_computed} are the values present in the diagonal of the $4 \times 4$ matrix $W$ from the SVD.

Given that the $W_{diag_{d1}}$ has four entries, then the rank of $A$ is four. 
Thus, given $W_{diag_{d1}}$ is a $4 \times 4$ matrix of rank four, there are no linear dependencies between $x_0$, $y_0$, $a$ and $b$ for the first dataset (d1).

The second dataset (d2), however, generates a $W_{diag_{d2}}$, where the fourth value for $W$ is considerably smaller than that of the others.
The rank of the matrix $A_{d2}$ can hence be considered to be three.
There are therefore linear dependency between two of the parameters in the second dataset.


