\subsection{Mathematical Model}
The two equations \ref{eq:u_x} and \ref{eq:v_x} where given.

%The linear equation in equation \ref{eq:mathematicalModel} is used to solve the two equations for $(v_i)_{i=0}^N$ and $(u_i)_{i=0}^N$.

\begin{eqnarray}
u(x) &=& c_1 + k_1 \int^{\omega/2}_{-\omega/2} F(x,y,d)v(y) \partial y \qquad \because c_1 = \varepsilon_1  \sigma T_1^4, k_1 = 1 - \varepsilon_1 \label{eq:u_x} \\
v(y) &=& c_2 + k_2 \int^{\omega/2}_{-\omega/2} F(x,y,d)u(x) \partial x \qquad \because c_2 = \varepsilon_2  \sigma T_2^4, k_2 = 1 - \varepsilon_2 \label{eq:v_x}
\end{eqnarray}

These can be rewritten with help of the trapezoidal rule to give the equations \ref{eq:u_trap} and \ref{eq:v_trap}.
Where $(v_i)_{i=0}^N$ and $(u_i)_{i=0}^N$ are the values for the two functions in the points $(x_i)_{i=0}^N = -\omega/2 + i h$ used in the trapezoidal equation and $h = \omega/N$.

\begin{eqnarray}
u_j &=& c_1 + k_1 h \left[ \frac{F(x_j, x_0, d) \cdot v_0 + F(x_j, x_N, d) \cdot v_N}{2} + \sum_{i = 1}^{N-1} \left( F(x_j, x_i, d) \cdot v_i \right) \right] \label{eq:u_trap}\\
v_j &=& c_2 + k_2 h \left[ \frac{F(x_0, x_j, d) \cdot u_0 + F(x_N, x_j, d) \cdot u_N}{2} + \sum_{i = 1}^{N-1} \left( F(x_i, x_j, d) \cdot u_i \right) \right] \label{eq:v_trap}
\end{eqnarray}

This gives the linear equations seen in ...

\begin{eqnarray}
c_1 &=& u_j - k_1 h \left[ \frac{F(x_j, x_0, d) \cdot v_0 + F(x_j, x_N, d) \cdot v_N}{2} + \sum_{i = 1}^{N-1} \left( F(x_j, x_i, d) \cdot v_i \right) \right] \label{eq:u_trap_lin}\\
c_2 &=& v_j - k_2 h \left[ \frac{F(x_0, x_j, d) \cdot u_0 + F(x_N, x_j, d) \cdot u_N}{2} + \sum_{i = 1}^{N-1} \left( F(x_i, x_j, d) \cdot u_i \right) \right] \label{eq:v_trap_lin}
\end{eqnarray}


The equations \ref{eq:u_trap_lin} and \ref{eq:v_trap_lin} can then be written into matrix form of equation \ref{eq:equationFormat}, where $A$, $x$ and $b$ are given in equations \ref{eq:A_matrix} and \ref{eq:c-b_matrix}.

\begin{equation}
A \cdot x = b
\label{eq:equationFormat}
\end{equation}

\begin{landscape}

\begin{equation}
% -- A --
A = 
\left[
\begin{array}{*{5}{c}:*{5}{c}}
\multicolumn{5}{c}{\multirow{5}{*}{\Huge{$I$}}} 
& \frac{G(x_0,x_0)}{2}
& G(x_0,x_1)
& \cdots
& G(x_0,x_{N-1})
& \frac{G(x_0,x_N)}{2} \\

&&&&& \frac{G(x_1,x_0)}{2}
& G(x_1,x_1) 
& \cdots
& G(x_{1},x_{N-1}) 
& \frac{G(x_{1},x_{N})}{2} \\

&&&&& \vdots
& \vdots
& \ddots
& \vdots
& \vdots\\

&&&&& \frac{G(x_{N-1},x_0)}{2}
& G(x_{N-1},x_{1}) 
& \cdots
& G(x_{N-1},x_{N-1}) 
& \frac{G(x_{N-1},x_{N})}{2} \\

&&&&& \frac{G(x_N,x_0)}{2} 
& G(x_{N},x_{1}) 
& \cdots
& G(x_{N},x_{N-1})
& \frac{G(x_{N},x_{N})}{2} \\
 \hdashline
 
\frac{H(x_0,x_0)}{2}
& H(x_1,x_0)
& \cdots
& H(x_{N-1},x_0)
& \frac{H(x_N,x_0)}{2} &
\multicolumn{5}{c}{\multirow{5}{*}{\Huge{$I$}}}
\\

\frac{H(x_0,x_1)}{2}
& H(x_1,x_1) 
& \cdots
& H(x_{N-1},x_1) 
& \frac{H(x_N,x_1)}{2} \\

\vdots
& \vdots
& \ddots
& \vdots
& \vdots\\

\frac{H(x_0,x_{N-1})}{2}
& H(x_1,x_{N-1}) 
& \cdots
& H(x_{N-1},x_{N-1}) 
& \frac{H(x_N,x_{N-1})}{2} \\

\frac{H(x_0,x_N)}{2} 
& H(x_1,x_N) 
& \cdots
& H(x_{N-1},x_{N})
& \frac{H(x_{N},x_{N})}{2} \\
\end{array}
\right]
\label{eq:A_matrix}
\end{equation}

Where $G$ and $H$ are given in equation \ref{eq:G_explained} and \ref{eq:H_explained} and $I$ is the identity matrix (in this case of size $(N+1) \times (N+1)).$

\begin{eqnarray}
G(x,y) &=&  -k_1 h F(x,y,1) \label{eq:G_explained} \qquad \because d = 1 \\
H(x,y) &=&  -k_2 h F(x,y,1)\label{eq:H_explained}
\end{eqnarray}

\end{landscape}

\begin{equation}
% -- x --
x = 
\left[
\begin{array}{c}
u_0 \\
u_1 \\
\vdots \\
u_N \\
v_0 \\
v_1 \\
\vdots \\
v_N \\
\end{array}
\right]
\qquad
% -- b --
b =
\left[
\begin{array}{c}
c_1 \\
c_1 \\
\vdots \\
c_1 \\
c_2 \\
c_2 \\
\vdots\\
c_2 \\
\end{array}
\right]
\label{eq:c-b_matrix}
\end{equation}


