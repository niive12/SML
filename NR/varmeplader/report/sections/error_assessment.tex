\section{Error assessment}

An estimate of the error can be calculated using the Richardson extrapolation.
Since the integration converges to a more accurate estimate with a higher number of iterations the error can be calculated using the last two results to see how well it converges.
The current integral is written as \(X_{h3}\) and the previous is thus called \(X_{h2}\) and the result before that is called \(X_{h1}\).
The order of a trapezoidal integration will always be 2 since the division is halfway between two points.
% If the order is unknown then it can be calculated using equation \ref{eq:REorder}.
Then the error can be estimated using equation \ref{eq:REerr}.

% \begin{equation}
%     \alpha^k = \frac{Q1_{h1}-Q1_{h2}}{Q1_{h2}-Q1_{h3}};
%     \label{eq:REorder}
% \end{equation}

\begin{equation}
    \text{error} = \frac{Q1_{h2}-Q1_{h1}}{\alpha^{k}-1};
    \label{eq:REerr}
\end{equation}

In table \ref{tb:error_estimate} is the integral and error shown for \(Q1\) and \(Q2\).

\begin{table}[H]
\centering
\begin{tabular}{*{5}{|c}|}  \hline
  N&                 Q1&                     Q2&                      error1&                 error2\\ \hline
  4&    1263.0042994788&        -299.1212789388&               -6.1320964286&         -12.5601797558\\ \hline
  8 &    1261.3382974160 &        -302.4650240096 &               -2.1290072849 &          -4.2929309229\\ \hline
 16 &    1260.9180057949 &        -303.3075790665 &               -0.5553340210 &          -1.1145816903\\ \hline
 32 &    1260.8127006578 &        -303.5186206685 &               -0.1400972070 &          -0.2808516856\\ \hline
 64 &    1260.7863599278 &        -303.5714060874 &               -0.0351017124 &          -0.0703472007\\ \hline
128 &    1260.7797738435 &        -303.5846040033 &               -0.0087802433 &          -0.0175951396\\ \hline
256 &    1260.7781272661 &        -303.5879035798 &               -0.0021953614 &          -0.0043993053\\ \hline
\end{tabular}
\caption{Error estimates.}
\label{tb:error_estimate}
\end{table}
