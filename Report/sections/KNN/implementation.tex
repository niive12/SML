\subsection{K-NN Implementation}
The K-NN algorithm is implemented as in code \ref{code:KNN_implementation}.
The implementation of how the result of the voting is calculated, is not shown here.


\lstinputlisting[language = R,
firstnumber = 291,
firstline = 291, 
lastline = 311, 
captionpos=b,
caption = {K-NN implementation.},
label = {code:KNN_implementation}]{../Code/KNN/01/test.R}


The function takes three arguments, \texttt{testVector}, \texttt{trainVectors} and \texttt{k}. 
\texttt{testVector} is a vector containing the set of pixel values of the character that is wished identified. 
\texttt{trainVectors} is a three dimensional set of vectors. 
This makes it possible to distinguish between the character represented by the group and the individual characters and their data.

Line 290 to 293, in code \ref{code:KNN_implementation}, defines the two arrays \texttt{neighbours} and \texttt{testDistance}.
These are used to store the \texttt{k} nearest neighbours and calculate the distance between two characters respectively.

The code then, in line 295 to 299, loops through all the different character groups and the characters within.
For each character the distance is then calculated to the \texttt{testVector}.
If the distance is smaller than that of one of the previously inserted into \texttt{neighbours}, 
then lines 301 to 309 insert it into the \texttt{neighbours} list in sorted order and the one furthest away is removed.

Once this is done, the \texttt{neighbours} list can then be inspected to guess which character it is.