% \subsection{K-NN Theory}
% The K-Nearest Neighbour (K-NN) method is in this report used to classify a specific unknown character to a set of characters ranging from 0 to 9.
% The method utilizes the euclidean distance to the nearest characters in the training set.

% Knowing the K nearest neighbours to a unknown character, then the character can be classified depending on the surrounding pixels.
% This is done by taking the most often occurring character within the K-NN and assuming this is also the same character.

The K-Nearest Neighbour (K-NN) method is in this report used to classify a unknown digit to a set of known digits ranging from zero to nine.
The data is split into two groups: One is for training and one is for testing. 
The nearest neighbour is found using the euclidean distance to the pixel values in a training set with known values.
The result is a vote between the K closest neighbours.

To see how this method performs a multitude of tests are performed to see how multiple parameters affect the success rate and the speed.
