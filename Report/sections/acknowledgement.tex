\section*{Abstract}
This report considers the problem of handwritten digit recognition.
The data of in total twenty people are used to test and compare the performance of the K-NN and Decision Trees for the given problem.

Data preprocessing is considered first and how to optimize the result for the two algorithms separately.
This is completed to achieve the best possible results concerning both time consumption and success rate of the algorithms.
The methods used for preprocessing are smoothing, z-score and principle component analysis (PCA).

Hereafter, the two algorithms are compared to and the worst handwritten dataset is found.
This was found testing each datasets success rate using the remaining datasets.

It was from these test found that the K-NN algorithm performs best when applying PCA to the smoothed and z-score normalized data.
Furthermore the decision trees perform the best when using data which only has been smoothed.
It is also found that the worst handwriting belonged to the dataset G2M1.


\section*{Acknowledgement}
This piece is written as a report by Nikolaj Iversen and Lukas Schwartz.
The report was split such that Nikolaj was mainly in charge of the concerning decision trees (section \ref{sec:tree}) and Lukas that of K-NN (section \ref{sec:KNN}).
Other tasks where split as seemed fit or written in collaboration.