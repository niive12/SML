This section considers the use of the K-NN algorithm when the datasets become very large.
First the pure K-NN algorithm is considered on a big dataset.
Then Principle Component Analysis is considered to see how this can decrease the runtime of the K-NN prediction.
After this the success of the K-NN algorithm will be compared when using normalization on the data and smoothing.
Finally K-means will be considered in simplification of the data.

Because of the large amount of data, only the 100 DPI pictures will be used.