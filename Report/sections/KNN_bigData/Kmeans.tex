\subsection{K-means}
K-means is used to group the dataset into categories depending on their location in the x-dimensional space determined by the amount of features representing one of them.
The center of these clusters are then computed and identified as a specific category.
Using these centres, a unknown element can then be categorized depending on which cluster is the closest.

To find the number of clusters best representing the dataset, then the within-group heterogeneity and homogeneity was computed for various K's as seen on figure \textbf{ref!!}.
The elbow point can from here be determined to be at \textbf{K = something}.


\textbf{Plot within-group heterogeneity and homogeneity}


\textbf{Plot time taken to predict elements using K-means}

