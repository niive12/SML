\subsection{Smoothing}


\begin{figure}[H]
\centering
\missingfigure{effect of smoothing on image}
\end{figure}

Applying a smoothing function can give the image an advantage when using the nearest neighbour analysis.
By taking an average of the neighbouring pixels the lines in the digit should be wider and the digits should have a better chance of overlapping.
The danger is that too much smoothing could make the whole image one colour and would completely destroy any chance of analysis.
A Gaussian distribution (also referred to as a normal distribution) a naturally occurring distribution that happens when a random result occurs around a mean. 
The $\sigma$ signifies the deviation of the distribution. 
The 2D equation of a Gaussian distribution is shown in equation \ref{eq:gauss}. 

\begin{equation}
G(x,y) = \frac{1}{2\pi \sigma^2} e^{- \frac{x^2+y^2}{2\sigma^2}} \label{eq:gauss}
\end{equation}

Using the Gaussian filter to smooth an image will weigh the distance to the pixel.
A small $\sigma$ will make a small deviation and thus heavily weigh the center pixel.
The results of the filtering methods were also compared to the raw image with 100, 200 and 300 DPI.
These results are compared with an averaging filter (avg) which takes the average of the four neighbouring pixels and a Gaussian filter (G) with different values for sigma.
These tests were done 10 times, using cross validation, and the mean of each success rate is plotted in figure \ref{fig:smooth}. 
Since the variance is too small to be seen in the figure the mean and variance is shown in table \ref{tb:smooth}.
The averaging filter does not give a measurable different result from not using a filter.
The Gaussian filter does improves the success rate for some values of sigma, but a larger $\sigma$ makes it worse.

