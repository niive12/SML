\section{Introduction}

The classification of handwritten characters is used in a wide range of products to day.
Hence, this report goes in depth with how the numbers from zero to nine can be classified using machine learning algorithms.

The data set consists of a set of handwritten characters from zero to nine.
These were constructed by the students enrolled in the course Statistical Machine learning (RM-SML-E1) in 2015 at the University of Southern Denmark (SDU).
The characters were written in boxes of $0.55 \times 0.55 \textit{ cm}$ on a sheet with $20 \times 20$ boxes for each character.
The set used in this report is the 100DPI dataset.
Each number is hence stored as a $20 \times 20 \textit{ pixel}$ matrix containing the handwritten character.

The methods used for classification are K-Nearest Neighbours and Decision Trees.
This is to compare a method of lazy supervised learning against a method of supervised learning.
Furthermore a set of different ways to pre-process the data is explored.
Finally the two methods are compared with each at the best parameters and preprocessing settings.

The goal is to tests the handwritten digits from 20 people enrolled in the course.
The first test is where all people are mixed together, everybody contributing 90\% of their data to the training set and 10\% to the test set.
This is considered the easy problem as special ways of writing a digit will be represented in the training data.

Another test is to use 19 people's digits as training data and use the last person to test.
This is considered the hard problem.

To test the performance of the classification methods a simplified problem has been constructed.
It takes training data from a single person, Lukas Schwartz, which is referred to as Group 3 member 2 (G3M2).
By splitting the data 90\% for training and 10\% for testing
360 digits from each class as training data and 40 digits was used.
Testing each parameter with the simplified problem means the parameters could be tested in less time than if using the entire data set.
