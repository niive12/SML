\section{Introduction}

The classification of handwritten characters is used in a wide range of products to day.
Hence, this report goes in depth with how the numbers from zero to nine can be classified using machine learning algorithms.

The data set consists of a set of handwritten characters from zero to nine.
These were constructed by the students enrolled in the course Statistical Machine learning (RM-SML-E1) in 2015 at the University of Southern Denmark (SDU).
The characters were written in boxes of $0.575 \times 0.575 \textit{ mm}$ on a sheet with $20 \times 20$ boxes for each character.
The set used in this report is the 100DPI dataset.
Each number is hence stored as a $20 \times 20 \textit{ pixel}$ matrix containing the handwritten character.

The methods used for classification are K-Nearest Neighbours and Decision Trees.
Furthermore a set of different ways to pre-process the data is explored.
Finally the two methods are compared with each at the best parameters and preprocessing settings.



